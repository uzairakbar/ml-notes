%%%%%%%%%%%%%%%%%%%%%%%%%%%%%%%%%%%%%%%%%%%%%%%%%%%%%%%%%%%%%%%%%%%
%
% Ce gabarit peu servir autant les philosophes que les scientifiques ; 
% et même d'autres genres, vous en faites ce que vous voulez.
% J'ai modifié et partagé ce gabarit afin d'épargner à d'autres 
% d'interminables heures à modifier des gabarits d'articles anglais. 
% 
% L'ajout d'une table des matières et une bibliographie a été ajoutée,
% rendant le gabarit plus ajusté aux besoins de plusieurs.
%
% Pour retrouvé le gabarit original, veuillez télécharger les
% documents suivants: llncs2e.zip (.cls et autres) et 
% typeinst.zip (.tex). Les documents ci-haut mentionnés ne sont pas 
% disponibles au même endroit, alors je vous invite à fouiller le web. 
%
% Pour l'instant (02-2016) ils sont disponibles tous deux ici :
%
% http://kawahara.ca/springer-lncs-latex-template/
%
% Netkompt
%
%%%%%%%%%%%%%%%%%%%%%%%%%%%%%%%%%%%%%%%%%%%%%%%%%%%%%%%%%%%%%%%%%%%


%%%%%%%%%%%%%%%%%%%%%%% file typeinst.tex %%%%%%%%%%%%%%%%%%%%%%%%%
%
% This is the LaTeX source for the instructions to authors using
% the LaTeX document class 'llncs.cls' for contributions to
% the Lecture Notes in Computer Sciences series.
% http://www.springer.com/lncs       Springer Heidelberg 2006/05/04
%
% It may be used as a template for your own input - copy it
% to a new file with a new name and use it as the basis
% for your article.
%
% NB: the document class 'llncs' has its own and detailed documentation, see
% ftp://ftp.springer.de/data/pubftp/pub/tex/latex/llncs/latex2e/llncsdoc.pdf
%
%%%%%%%%%%%%%%%%%%%%%%%%%%%%%%%%%%%%%%%%%%%%%%%%%%%%%%%%%%%%%%%%%%%

\documentclass[runningheads,a4paper]{llncs}

\usepackage[utf8]{inputenc}

%\usepackage{natbib}
%\bibliographystyle{IEEEtran}
%\bibliographystyle{apacite}

\usepackage{amssymb}
\setcounter{tocdepth}{2}
\usepackage{graphicx}
\usepackage{amsmath}
\DeclareMathOperator*{\argmin}{arg\,min}

%////////////////////////////////////////////////////////
% My packages
\usepackage{hanging}
\usepackage{caption}
%\usepackage[utf8]{inputenc}
%\usepackage[english]{babel}
\usepackage[style=apa,citestyle=apa]{biblatex}
\addbibresource{references.bib}
%\bibliography{references}
\usepackage{etoolbox}
\patchcmd{\thebibliography}{\section*}{\section}{}{}

\usepackage{color}
\usepackage[normalem]{ulem}
\usepackage{url}

\usepackage{hyperref}
\hypersetup{
%    colorlinks=true,
%    linkcolor=blue,
    filecolor=magenta,      
    urlcolor=cyan,
}

\DeclareUrlCommand\ULurl{%
  \renewcommand\UrlFont{\ttfamily\color{blue}}%
  \renewcommand\UrlLeft{\uline\bgroup}%
  \renewcommand\UrlRight{\egroup}}
%////////////////////////////////////////////////////////



%%%%%%%%%%%%%%%%%%%%
\usepackage{mathtools}
%\usepackage[bottom]{footmisc}
%\usepackage{subcaption}
% end - My packages
\usepackage{float}
%\usepackage[french]{babel} % Pour adopter les règles de typographie française
\usepackage[T1]{fontenc} % Pour que les lettres accentuées soient reconnues
\usepackage{lastpage}

\usepackage{url}
% \urldef{\mailsa}\path|{alfred.hofmann, ursula.barth, ingrid.haas, frank.holzwarth,|
% \urldef{\mailsb}\path|anna.kramer, leonie.kunz, christine.reiss, nicole.sator,|
% \urldef{\mailsc}\path|erika.siebert-cole, peter.strasser, lncs}@springer.com|    
% \newcommand{\keywords}[1]{\par\addvspace\baselineskip
% \noindent\keywordname\enspace\ignorespaces#1}
% \addtocounter{page}{1}
\begin{document}


\mainmatter 

\title{Machine Learning Notes}
\titlerunning{ML Notes}
\author{by Uzair Akbar}
\institute{}
\authorrunning{ML Notes}
%\toctitle{Table of Contents}
%\tocauthor{{}}
% Made changes here for TOC title line
{\def\addcontentsline#1#2#3{}\maketitle}
\medskip
\begingroup
\let\clearpage\relax
\tableofcontents
\endgroup
\medskip
\medskip

\section{Linear Algebra}

\section{Prob. \& Stats.}

\section{Convex Optimization}

\appendix
\section{Miscellaneous}
\subsection{Topics To Read}
\subsection{Resources}


% \section{Task 1 (Exponential Cost Function)}
% \subsection{Part a)}

% \subsubsection{Definitions:}
% \begin{equation*}
%      \begin{aligned}
%      &\pi^{k}:={\mu_{k}, \mu_{k+1}, \dots, \mu_{n-1}}
%      \\
%      &J_{k}^{*}(x_{k}):=\min_{\pi^k} \mathop{\mathbb{E}}_{\substack{w_i\\i=k,\dots,N-1}}\Bigg(\exp\bigg(g_N(x_N)+\sum_{i=k}^{N-1}{g_i(x_i,\mu_i,w_i)}\bigg)\Bigg)
%      \\
%      &J_{k}^{*}(x_{k}):=\exp\big(g_N(x_N)\big)
%      \end{aligned}
% \end{equation*}

% \subsubsection{To Prove:} To show that $J_k^*$ is equal to $J_k$, where $J_k$ is given by

% \begin{equation}
% \left.
% \begin{aligned}
% &J_N(x_N)=\exp\big(g_N(x_N)\big)
% \\
% &J_k(x_k)=\min_{u_k \in U_k(x_k)}\mathop{\mathbb{E}}_{w_k}\bigg\{J_{k+1}\big(f_k(x_k,u_k,w_k)\big) \times \exp\big(g_k(x_k,u_k,w_k)\big)\bigg\}.
% \end{aligned}
% \right\}
% \end{equation}

% \subsubsection{Proof:}
% By definition, for $k=N$,
% \begin{equation*}
% J_N^*(x_N)=\exp\big(g_N(x_N)\big)=J_N(x_N).
% \end{equation*}

% Now, assume that

% \begin{equation*}
% J_{k+1}^*(x_{k+1})=J_{k+1}(x_{k+1}).
% \end{equation*}

% Now,
% \begin{equation*}
%     \begin{aligned}
%     \because\quad\;\;\;\pi^k&=\{\mu_k, \pi^{k+1}\}
%     \\
%     J_k^*(x_k)&=\min_{\pi^k} \mathop{\mathbb{E}}_{\substack{w_i\\i=k,\dots,N-1}}\Bigg(\exp\bigg(g_N(x_N)+g_k(x_k, \mu_k, w_k)+\sum_{i=k+1}^{N-1}{g_i(x_i,\mu_i,w_i)}\bigg)\Bigg)
%     \\
%     &=\min_{\mu^k}\mathop{\mathbb{E}}_{w_k}\bigg(\exp\big(g_k(x_k,\mu_k,w_k)\big) \times \quad\quad\quad\quad\quad\quad \text{( Principle of optimality )}\\
%     &\quad\quad\min_{\pi^{k+1}}\mathop{\mathbb{E}}_{\substack{w_i\\i=k+1,\dots,N-1}}\Big(\exp\big(g_N(x_N)+\sum_{i=k+1}^{N-1}{g_i(x_i,\mu_i,w_i)}\big)\Big)\bigg)
%     \\
%     &=\min_{\mu^k}\mathop{\mathbb{E}}_{w_k}\bigg(\exp\big(g_k(x_k,\mu_k,w_k)\big) \times J_{k+1}^*(x_{k+1})\bigg)
%     \\
%     &=\min_{u_k \in U_k(x_k)}\mathop{\mathbb{E}}_{w_k}\bigg(\exp\big(g_k(x_k,u_k,w_k)\big) \times J_{k+1}(x_{k+1})\bigg)
%     \\
%     \therefore J_k^*(x_k)&=J_k(x_k)
%     \end{aligned}
% \end{equation*}

% %-----------------------------------
% %-----------------------------------

% \subsection{Part b)}

% \subsubsection{Definitions:}
% \begin{equation*}
%     \begin{aligned}
%     &V_k(x_k)=\ln{J_k(x_k)}
%     \\
%     &g_k = g_k(x_k, u_k)
%     \end{aligned}
% \end{equation*}

% \subsubsection{To Prove:}
% To show that algorithm (3) can be rewritten as
% \begin{equation*}
%     \begin{aligned}
%     &V_N(x_N)=g_N(x_N),
%     \\
%     &V_k(x_k) = \min_{u_k \in U_k(x_k)}\bigg\{g_k(x_k, u_k)+\ln{\mathop{\mathbb{E}}_{w_k}}\Big\{\exp\Big(V_{k+1}\big(f_k(x_k, u_k, w_k)\big)\Big)\Big\}\bigg\}
%     \end{aligned}
% \end{equation*}

% \subsubsection{Proof:}

% \begin{equation*}
%     \begin{aligned}
%     &\because&V_k(x_k)&=&\ln{J_k(x_k)}\;\,&
%     \\
%     &\therefore&V_N(x_N)&=&\ln{J_N(x_N)}&=\ln{\Big(\exp\big(g_{N}(x_N)\big)\Big)}=g_N(x_N)
%     \\
%     \end{aligned}
% \end{equation*}

% Now, as $\ln$ is monotonically increasing, we can interchange it with $\min$.

% \begin{equation*}
%     \begin{aligned}
%     \because V_k(x_k)&=\ln{J_k(x_k)}
%     \\
%     &=\ln\bigg(\min_{u_k \in U_k(x_k)}\mathop{\mathbb{E}}_{w_k}\Big(J_{k+1}(x_{k+1}).\exp\big(g_k(x_k, u_k)\big)\Big)\bigg)
%     \\
%     &=\min_{u_k \in U_k(x_k)}{\ln{\bigg(\exp\big(g_k(x_k, u_k)\big).\mathop{\mathbb{E}}_{w_k}\Big(J_{k+1}(x_{k+1})\Big)\bigg)}}
%     \\
%     &=\min_{u_k \in U_k(x_k)}{{\Bigg(g_k(x_k, u_k)+\ln\bigg(\mathop{\mathbb{E}}_{w_k}\Big(J_{k+1}(x_{k+1})\Big)\bigg)\Bigg)}}
%     \\
%     \therefore V_k(x_k)&= \min_{u_k \in U_k(x_k)}\bigg\{g_k(x_k, u_k)+\ln{\mathop{\mathbb{E}}_{w_k}}\Big\{\exp\Big(V_{k+1}\big(f_k(x_k, u_k, w_k)\big)\Big)\Big\}\bigg\}
%     \end{aligned}
% \end{equation*}

%-----------------------------------
%-----------------------------------

\end{document}
